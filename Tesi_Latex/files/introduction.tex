\chapter{Introduzione}\label{ch:introduzione}
\section*{Motivazioni}
In Italia e nel mondo lo sport, il calcio in particolare, ha una grande importanza sociale e culturale.
\\A causa di questa importanza e del numero di appassionati, lo sport professionistico è un settore in cui vengono effettuati grandi investimenti e vengono spesi annualmente miliardi di euro, ad esempio i diritti televisivi di serie A per il triennio 2018-2021 sono stati venduti dalla Lega Calcio a oltre \textbf{1.4 miliardi} di euro l'anno. \cite{DirittiTriennio2018-21}
\\Non sorprenderà quindi che una buona percentuale dei soldi che entrano nelle casse dei club calcistici è data prorio dai sopracitati diritti, sempre in serie A nel 2015/2016 tale percentuale era mediamente il \textbf{48\%}. \cite{impattoDirittiTv}
\\I broadcaster televisivi, spendendo per tali diritti così tanti soldi e mirando ovviamente a effettuare del profitto, devono avere un gran numero di abbonati. Per raggiungere tale obbiettivo essi offrono servizi di qualità, come ad esempio studi pre/post partita, approfondimenti, ma sopratutto \textbf{highlights} che vanno a riassumere i momenti salienti di una partita. 
\\La creazione di quest'ultimi può sembrare un'operazione banale e infatti lo è, tuttavia richiede che una persona annoti manualmente le azioni pericolose e in particolare i goal.
\\Questo procedimento dovendo essere ripetuto per tutte le partite rappresenta un costo non indifferente per i suddetti broadcaster, che avrebbero un grande interesse ad automatizzare il procedimento.
\section*{Presentazione del lavoro}
In questo elaborato di tesi, partendo dal lavoro di \citet{soccerNet} , verrà descritta una tecnica di automatizzazione del problema citato in precedenza, la quale tuttavia non genera veri e propri highlights, i quali includerebbero oltre ai goal anche azioni pericolose e momenti salienti della partita, ma isola durante l'arco di una partita tre tipi di azione: i \textbf{cartellini}, le \textbf{sostituzioni} ed i \textbf{goal}.
\\Il lavoro descritto è suddiviso in due parti, la prima parte consiste in un problema di \textbf{classificazione} in cui si è proceduto ad addestrare una rete neurale, nella seconda è stata usata la medesima rete neurale precedentemente addestrata per isolare le azioni di nostro interesse.
\\Il Dataset su cui abbiamo lavorato è formato da \textbf{500} partite prese dai maggiori campionati europei.


